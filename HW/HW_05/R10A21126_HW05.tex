\documentclass[UTF8,a4paper,10pt]{article}

% \begin{equation*}
%   \begin{aligned}
%   \end{aligned}
% \end{equation*}

% \begin{mybox}{}
% \end{mybox}


% \begin{Problem}[]{}
% \end{Problem}

% \begin{solution}\,
% \end{solution}
  

% \begin{enumerate}[label=(\alph*)]
% \end{enumerate} 

% \setcounter{section}{3} 
% \setcounter{theorem}{3}

% \begin{theorem}\label{thm:3.4}
%   If $E = \bigsqcup_{k} E_k$ is a countable union of sets, then $|E|_e \leq \sum_{k} |E_k|_e$.
%   \end{theorem}

%  \footcite[][42]{Wheeden_Zygmund_2015}

% \documentclass[UTF8,a4paper,14pt]{article}
% \usepackage[utf8]{inputenc}
\usepackage{amsmath}
% \usepackage{algorithm,algorithmic}
\usepackage[linesnumbered,ruled,vlined]{algorithm2e}

% \usepackage{algorithmicx}
% \usepackage{algpseudocode}
\usepackage{hyperref}

% \usepackage{algpseudocode}
\usepackage{amssymb}
\usepackage{amsfonts}
%for 字體
%https://tug.org/FontCatalogue/
% \usepackage[T1]{fontenc}
% \usepackage{tgbonum}
% \usepackage[bitstream-charter]{mathdesign}
% \usepackage[T1]{fontenc}
% \usepackage{bm}#粗體
%\usepackage{boondox-calo}
\usepackage{textcomp}
\usepackage{fancyhdr}%导入fancyhdrf包
\usepackage{ctex}%导入ctex包
\usepackage{enumitem} %for在Latex使用條列式清單
\usepackage{varwidth}
\usepackage{soul} %for \ul
\usepackage{comment}%\begin{comment}\end{comment}
\usepackage{cancel}%\cancel{}
%\usepackage{unicode-math}

\usepackage[dvipsnames, svgnames, x_11names]{xcolor}

\usepackage[low-sup]{subdepth}
\usepackage{subdepth}

\newcommand{\indep}{\Perp \!\!\! \Perp}

\usepackage{amsthm}
\DeclareMathOperator{\R}{\mathbb{R}}
\DeclareMathOperator{\E}{\mathbb{E}}
\DeclareMathOperator{\Var}{\textbf{Var}}
\DeclareMathOperator{\Cov}{\textbf{Cov}}
\DeclareMathOperator{\Cor}{\textbf{Cor}}
\DeclareMathOperator{\X}{\mathbf{X}}
\DeclareMathOperator{\Pro}{\mathbf{P}}
\DeclareMathOperator{\M}{\mathbf{M}}
\DeclareMathOperator{\Id}{\mathbf{I}}
\DeclareMathOperator{\Y}{\mathbf{Y}}
\DeclareMathOperator{\MSFE}{\mathbf{MSFE}}
\DeclareMathOperator{\e}{\mathbb{e}}
\DeclareMathOperator{\V}{\mathbf{V}} 
\DeclareMathOperator{\tr}{\text{tr}}
\DeclareMathOperator{\A}{\textbf{A}}
\DeclareMathOperator{\diag}{diag}
\DeclareRobustCommand{\rchi}{{\mathpalette\irchi\relax}}
\newcommand{\irchi}[2]{\raisebox{\depth}{$#1\chi$}} % inner command, used by \rchi
\DeclareMathOperator*{\argmax}{arg\,max}
\DeclareMathOperator*{\argmin}{arg\,min}

\DeclareMathSizes{20}{10}{10}{5}

\usepackage[a4paper, margin=1in]{geometry}
% \setlength\parskip{5ex}% it would be better define distance in ex (5ex) 
                         %  or in pt, pc, mm, etc (see edit below)

\setlength{\parindent}{0pt}
\usepackage{array, makecell} %


%中英文設定
%\usepackage{fontspec}
% \setmainfont{TeX Gyre Termes}
% \usepackage{xeCJK} %引用中文字的指令集
% %\setCJKmainfont{PMingLiU}
% \setCJKmainfont{DFKai-SB}





% \setmainfont{Times New Roman}
% \setCJKmonofont{DFKai-SB}
\pagenumbering{arabic}%设置页码格式
\pagestyle{fancy}
\fancyhead{} % 初始化页眉
\usepackage{advdate}

% \newcommand{\yesterday}{{\AdvanceDate[-1]\today}}

\fancyhead[C]{Real Analysis II \quad HW  04\quad  R10A21126\quad  WANG YIFAN\quad   \today}
%\fancyhead[LE]{\textsl{\rightmark}}
%\fancyfoot{} % 初始化页脚
%\fancyfoot[LO]{奇数页左页脚}
%\fancyfoot[LE]{偶数页左页脚}
%\fancyfoot[RO]{奇数页右页脚}
%\fancyfoot[RE]{偶数页右页脚}

% \title{{Econometrics HW 05}}
% \author{R10A21126}
% \date{\today}

%\fancyhf{}
\usepackage{lastpage}
\cfoot{Page \thepage \hspace{1pt} of\, \pageref{LastPage}}

\renewcommand{\headrulewidth}{0.1pt}%分隔线宽度4磅
%\renewcommand{\footrulewidth}{4pt}

\allowdisplaybreaks
\usepackage[english]{babel}
%\usepackage{amsthm}
\newtheorem{theorem}{Theorem}[section]
\newtheorem{corollary}{Corollary}[theorem]
\newtheorem{lemma}[theorem]{Lemma}


\usepackage[most]{tcolorbox}

\definecolor{babyblue}{rgb}{0.54, 0.81, 0.94}

\newtcolorbox[auto counter]{mybox}[1]{
  % Define a new tcolorbox style with custom paragraph spacing
  before upper={\parskip=10pt},
    after upper={\parskip=10pt},
    enhanced,
    arc= 1 mm,boxrule=1.5pt,
    colframe=babyblue!80!pink,
    colback=white,
    coltitle=black,
    % colback=blue!5!white,
    attach boxed title to top left=
    {xshift=1.5em,yshift=-\tcboxedtitleheight/2},
    boxed title style={size=small,
    % frame hidden,
    colback=White},
    top=0.15in,
    % fonttitle=\bfseries,
    title= {#1},
    breakable
  }

\newtcolorbox[auto counter]{Problem}[2][]{
    enhanced,drop shadow={Pink!50!white},
    colframe=pink!80!white,
    fonttitle=\bfseries,
    title=Problem ~\thetcbcounter. #2,
    %separator sign={.},
    coltitle=black,
    colback=pink!15,
    top=0.15in,
    breakable
  }

\newenvironment{solution}
  {\renewcommand\qedsymbol{$\blacksquare$}\begin{proof}[Solution]}
  {\end{proof}}

\theoremstyle{definition}
\newtheorem{definition}{Definition}[section]

%\theoremstyle{notation}
\newtheorem*{notation}{\underline{Notation}}
%\newtheorem*{convention}{\underline{Convention}}
\newtheorem*{convention}{\underline{Convention}}

\theoremstyle{remark}
\newtheorem*{remark}{Remark}

\newenvironment{amatrix}[2]{%% [2] for 2 parameters 
  \left[\begin{array}
    %{cc\,|\,cc}
    %  {@{}*{#2}{c}\,|\,c*{#1}{c}}
     {{}*{#1}{c}\,|\,c*{#2}{c}}
}{%
  \end{array}\right]
}
% For augmented matrix  
%https://tex.stackexchange.com/questions/2233/whats-the-best-way-make-an-augmented-coefficient-matrix


% defines the paragraph spacing
\setlength{\parskip}{0.5em}


\usepackage[sorting=none, citestyle=verbose-inote,backref=true,ibidtracker=context,mincrossrefs=99,backend=biber, 
url = false,
doi = false, isbn=false,]{biblatex}

\addbibresource{R10A21126.bib}

\usepackage{graphicx}
\graphicspath{ {images/} }
\usepackage{caption}

% global change
\SetKwInput{KwData}{Input}
\SetKwInput{KwResult}{Output}
% https://tex.stackexchange.com/questions/299771/how-do-i-rename-data-from-kwdata-and-result-from-kwresult-in-begi

\hypersetup{hidelinks}

\let\oldemptyset\emptyset
\let\emptyset\varnothing
% https://tex.stackexchange.com/questions/22798/nice-looking-empty-set


\usepackage{pgfplots}

\begin{document}


\begin{mybox}{H\"older's Inequality}

    \textit{If $1 \leq p \leq \infty$ and $\frac{1}{p} + \frac{1}{p'} = 1$, then $\|fg\|_1 \leq \|f\|_p \|g\|_{p'}$; that is,}

    \begin{equation*}
        \begin{aligned}
          \int_E |fg| &\leq \left( \int_E |f|^p \right)^{1/p} \left( \int_E |g|^{p'} \right)^{1/p'}, \quad 1 < p < \infty; \\
          \int_E |fg| &\leq \left( \text{ess sup}_E |f| \right) \int_E |g|.
        \end{aligned}
      \end{equation*} 
      
    

\end{mybox}


\begin{equation*}
    \begin{aligned}
        \|fg\|_r &= \left( \int |fg|^r \right)^{1/r}\\
        \|f g\|_r^r &= \left( \int |fg|^r \right)^{1} = \|f^r g^r\|_1
    \end{aligned}
  \end{equation*}        


\begin{Problem}[]{}
    \begin{itemize}
        \item[(a)] Let $1 \leq p_i, r \leq \infty$ and $\frac{1}{p_1} + \cdots + \frac{1}{p_k} = \frac{1}{r}$. Prove the following generalization of Hölder’s inequality:
        \[
        \| f_1 \cdot f_2 \cdots f_k \|_r \leq \| f_1 \|_{p_1} \cdot \| f_2 \|_{p_2} \cdots \| f_k \|_{p_k}
        \]
        \item[(b)] Let $1 \leq p < r < q \leq \infty$ and define $\theta \in (0, 1)$ by
        \[
        \frac{1}{r} = \frac{\theta}{p} + \frac{1 - \theta}{q}
        \]
        Prove the interpolation estimate:
        \[
        \| f \|_r \leq \| f \|_p^\theta \| f \|_q^{1-\theta}
        \]
    \end{itemize}
    

\end{Problem}



\begin{solution}\,

    \subsection*{(a) Generalization of Hölder's Inequality}


    We prove this by induction. 
    
    The \( k = 2 \) case is a consequence of H\"older's inequality:

    If \( \frac{1}{p_1} + \frac{1}{p_2} = \frac{1}{r} \), then \( \frac{r}{p_1} + \frac{r}{p_2} = 1 \), so
\begin{equation*}
  \begin{aligned}
    \|fg\|_r^r &= \|f^r g^r\|_1 \leq \|f^r\|_{p_1/r} \|g^r\|_{p_2/r} = \|f\|_{p_1}^r \|g\|_{p_2}^r.
  \end{aligned}
\end{equation*}

It is implied that
\begin{equation*}
    \begin{aligned}
      \|fg\|_r &\leq \|f\|_{p_1}\|g\|_{p_2}.
    \end{aligned}
  \end{equation*}

Now if \( \frac{1}{p_1} + \cdots + \frac{1}{p_k} = \frac{1}{r} \) for \( k > 2 \), we have
\begin{equation*}
  \begin{aligned}
    \|f_1 \cdots f_k\|_r &\leq \|f_1 \cdots f_{k-1}\|_s \|f_k\|_{p_k} \\
    &\leq \|f_1\|_{p_1} \cdots \|f_k\|_{p_k},
  \end{aligned}
\end{equation*}
where \( \frac{1}{s} = \frac{1}{r} - \frac{1}{p_k} = \frac{1}{p_1} + \cdots + \frac{1}{p_{k-1}} \).

\subsection*{(b) Interpolation Estimate}


Let $1 \leq p < r < q \leq \infty$ and define $\theta \in (0, 1)$ by
        \[
        \frac{1}{r} = \frac{\theta}{p} + \frac{1 - \theta}{q}
        \]




In other words, \(1/r\) is the convex interpolation between \(1/p\) and \(1/q\). 

\begin{equation*}
  \begin{aligned}
    \frac{1}{r} &= \frac{\theta}{p} + \frac{1 - \theta}{q}\\
    \frac{1}{r} &= \frac{1}{p/\theta} + \frac{1}{q/(1 - \theta)}
  \end{aligned}
\end{equation*}

Apply the Hölder’s Inequality,

\begin{equation*}
  \begin{aligned}
    \| f \|_r &= \| f^{\theta} f^{1-\theta} \|_r \\ \text{(Apply the Hölder’s Inequality)}\quad
    &\leq \| f^\theta \|_{p/\theta} \| f^{1-\theta} \|_{q/(1-\theta)}\\
     &=\| f \|_p^\theta \| f \|_q^{1-\theta}
  \end{aligned}.
\end{equation*}

The last equation is due to

\begin{equation*}
  \begin{aligned}
    \| f^{\theta} \|_{p/\theta} =  \left( \int |f^{\theta}|^{p/\theta} \right)^{\theta/p} = \left( \int |f|^{p} \right)^{\theta/p} = \| f \|_{p}^{\theta}
  \end{aligned}
\end{equation*}


  
\end{solution}



\begin{mybox}{}
  Lemma. (used in the proof of Problem 2)
  
   For \(a, b \in \mathbb{R}\), \(|a + b|^p \leq 2^p(|a|^p + |b|^p)\), where \(0 < p < \infty\).
  
  Proof.
  \begin{equation*}
  \begin{aligned}
  |a + b|^p &\leq (|a| + |b|)^p \\
  &\leq (2 \max\{|a|, |b|\})^p \\
  &= 2^p (\max\{|a|, |b|\})^p \\
  &\leq 2^p (|a|^p + |b|^p).
  \end{aligned}
  \end{equation*}
  
  \end{mybox}

\pagebreak


\begin{Problem}[]{}

    Let $f \in L^p(\mathbb{R}^n)$, where $0 < p < \infty$. Show that
\[
\lim_{Q \searrow x} \frac{1}{|Q|} \int_Q |f(y) - f(x)|^p \, dy = 0 \quad \text{a.e.}
\]

\end{Problem}




Let \(\{r_k\}\) be the rational numbers. First note that for any \(Q, x,\) and \(r_k\),

\begin{equation*}
  \begin{aligned}
   |f(y) - f(x)|^p \leq  |f(y) - r_k|^p +|r_k - f(x)|^p 
  \end{aligned}
  \end{equation*}
  
\begin{equation*}
\begin{aligned}
\frac{1}{|Q|} \int_Q |f(y) - f(x)|^p dy &\leq 2^p \frac{1}{|Q|} \int_Q |f(y) - r_k|^p dy + 2^p \frac{1}{|Q|} \int_Q |r_k - f(x)|^p dy \\
&= 2^p \frac{1}{|Q|} \int_Q |f(y) - r_k|^p dy + 2^p |r_k - f(x)|^p.
\end{aligned}
\end{equation*}

For every \(r_k\), let \(Z_k\) be the set in which the formula

\begin{equation*}
\lim_{Q \searrow  x} \frac{1}{|Q|} \int_Q |f(y) - r_k|^p dy = |f(x) - r_k|^p
\end{equation*}

is not valid. 

Since
\begin{equation*}
|f(y) - r_k|^p \leq 2^p (|f(y)|^p + |r_k|^p)
\end{equation*}

is locally integrable, by Lebesgue's Differentiation Theorem, \(|Z_k| = 0\). Let \(Z = \bigcup Z_k\), then \(|Z| = 0\).

Thus, if \(x \notin Z\), for every \(r_k\),

\begin{equation*}
  \begin{aligned}
\limsup_{Q \searrow x} \frac{1}{|Q|} \int_Q |f(y) - f(x)|^p dy &\leq 2^p |f(x) - r_k|^p + | r_k - f(x)|^p
\\
& = 2^{p+1} |f(x) - r_k|^p.
\end{aligned}
\end{equation*}

For an \(x\) at which \(f(x)\) is finite (in particular, almost everywhere since \(f \in L^p(\mathbb{R}^n)\)), by the density of rationals in \(\mathbb{R}^n\) we can choose \(r_k\) such that \(|f(x) - r_k|^p\) is arbitrarily small.

Thus
\begin{equation*}
\limsup_{Q \searrow x} \frac{1}{|Q|} \int_Q |f(y) - f(x)|^p dy = 0 \quad \text{a.e.}
\end{equation*}

and this completes the proof. Since 

\[\liminf_{Q \searrow x} \frac{1}{|Q|} \int_Q |f(y) - f(x)|^p dy 
\leq \limsup_{Q \searrow x} \frac{1}{|Q|} \int_Q |f(y) - f(x)|^p dy = 0 \quad \text{a.e.}
\]





% Let $f \in L^p(\mathbb{R}^n)$, where $0 < p < \infty$. We aim to show that
% \[
% \lim_{Q \searrow x} \frac{1}{|Q|} \int_Q |f(y) - f(x)|^p \, dy = 0 \quad \text{almost everywhere}.
% \]

% Here, $Q \searrow x$ means that $Q$ is a sequence of cubes shrinking to the point $x$.

% First, recall that since $f \in L^p(\mathbb{R}^n)$, $f$ is locally integrable. For almost every point $x \in \mathbb{R}^n$, the Lebesgue differentiation theorem states that
% \[
% \lim_{Q \searrow x} \frac{1}{|Q|} \int_Q |f(y) - f(x)| \, dy = 0.
% \]

% To prove the desired result, we need to generalize this to the $L^p$ norm. Consider the following steps:

% \begin{equation*}
% \begin{aligned}
%    &1. \text{Apply Hölder's Inequality:} \\
%    &\left( \frac{1}{|Q|} \int_Q |f(y) - f(x)|^p \, dy \right)^{\frac{1}{p}} \leq \left( \frac{1}{|Q|} \int_Q |f(y) - f(x)| \, dy \right)^{\frac{1}{p}}.
% \end{aligned}
% \end{equation*}

% \begin{equation*}
% \begin{aligned}
%    &2. \text{Use the Lebesgue Differentiation Theorem:} \\
%    &\lim_{Q \searrow x} \frac{1}{|Q|} \int_Q |f(y) - f(x)| \, dy = 0 \quad \text{for almost every } x \in \mathbb{R}^n.
% \end{aligned}
% \end{equation*}

% \begin{equation*}
% \begin{aligned}
%    &3. \text{Combine the Results:} \\
%    &\lim_{Q \searrow x} \left( \frac{1}{|Q|} \int_Q |f(y) - f(x)|^p \, dy \right)^{\frac{1}{p}} = 0.
% \end{aligned}
% \end{equation*}

% \begin{equation*}
% \begin{aligned}
%    &4. \text{Raise to the Power of } p: \\
%    &\lim_{Q \searrow x} \frac{1}{|Q|} \int_Q |f(y) - f(x)|^p \, dy = 0.
% \end{aligned}
% \end{equation*}

% Thus, we have shown that
% \[
% \lim_{Q \searrow x} \frac{1}{|Q|} \int_Q |f(y) - f(x)|^p \, dy = 0 \quad \text{almost everywhere}.
% \]


\pagebreak

\begin{Problem}[]{}

    Show that every subset \(\Lambda\) of a separable metric space \((M, d)\) is separable.

\end{Problem}


\begin{mybox}{separable}



\textbf{Def.} A \emph{metric space} \((X, d)\) is said to be \textbf{separable} if there exists a countable subset \(A \subseteq X\) that is dense in \(X\), i.e., \(\overline{A} = X\). 

That is, \((X, d)\) is \textbf{separable} if and only if there exists a countable subset \(A \subseteq X\) such that for  every \(x \in X\) and every \(\epsilon > 0\), there exists \(a \in A\) with \(d(x, a) < \epsilon\).

\end{mybox}


%  Let \( D = \{f_k\} \) be a countable dense set in \( M \).


%  i.e., 
%  \(\forall  \lambda\in\Lambda, \forall n\geq 1,\)
%  \(\exists f_k\in D\), s.t. \(d(\lambda, f_k )< 1/n\).
 
%  For \( n\geq 1 \), define \( D_n = \{ f \in D : \inf_{\lambda \in \Lambda} d(\lambda, f) < 1/n \} \). 
 
%  If \( f_k \in D_n \), pick \( \lambda_{k,n} \in \Lambda \) with \( d(\lambda_{k,n}, f_k) < 1/n \).
 
%  Claim: The subset \( \{\lambda_{k,n}\} \) is dense in \( \Lambda \).

%  ---

%  proof of Claim:

%  \(\forall  \lambda\in\Lambda, \forall n\geq 1, \exists \lambda_{k,n}\), s.t. 


% \begin{equation*}
% \begin{aligned}
% d(\lambda,\lambda_{k,n}) &\leq d(\lambda,f_k) + d(f_k,\lambda_{k,n})\\
% & 1/n +1/n.
% \end{aligned}
% \end{equation*}

% That is, \(d(\lambda,\lambda_{k,n})\to 0\), as \(n\to\infty\).

% This implies that  \( \{\lambda_{k,n}\} \) is dense in \(\Lambda\).







Let \( D = \{f_k\} \) be a countable dense set in \( M \).

i.e., 
\[
\forall \lambda \in \Lambda, \forall n \geq 1, \exists f_k \in D \text{ such that } d(\lambda, f_k) < \frac{1}{n}.
\]

For \( n \geq 1 \), define 
\[
D_n = \{ f \in D : \inf_{\lambda \in \Lambda} d(\lambda, f) < \frac{1}{n} \}.
\]

If \( f_k \in D_n \), pick \( \lambda_{k,n} \in \Lambda \) with 
\[
d(\lambda_{k,n}, f_k) < \frac{1}{n}.
\]

\begin{claim}[]{}
The subset \( \{\lambda_{k,n}\} \) is dense in \( \Lambda \).
\end{claim}

\begin{proof}
Consider any \( \lambda \in \Lambda \) and any \( n \geq 1 \). There exists \( \lambda_{k,n} \) such that
\[
d(\lambda, \lambda_{k,n}) \leq d(\lambda, f_k) + d(f_k, \lambda_{k,n}) = \frac{1}{n} + \frac{1}{n}.
\]
This simplifies to 
\[
d(\lambda, \lambda_{k,n}) \to 0 \text{ as } n \to \infty.
\]
This implies that every point in \( \Lambda \) can be approximated arbitrarily closely by elements of \( \{\lambda_{k,n}\} \), thereby proving that \( \{\lambda_{k,n}\} \) is dense in \( \Lambda \), which establishes the claim.
\end{proof}


\end{document}