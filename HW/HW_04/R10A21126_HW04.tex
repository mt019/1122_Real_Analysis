\documentclass[UTF8,a4paper,10pt]{article}

% \begin{equation*}
%   \begin{aligned}
%   \end{aligned}
% \end{equation*}

% \begin{mybox}{}
% \end{mybox}


% \begin{Problem}[]{}
% \end{Problem}

% \begin{solution}\,
% \end{solution}
  

% \begin{enumerate}[label=(\alph*)]
% \end{enumerate} 

% \setcounter{section}{3} 
% \setcounter{theorem}{3}

% \begin{theorem}\label{thm:3.4}
%   If $E = \bigsqcup_{k} E_k$ is a countable union of sets, then $|E|_e \leq \sum_{k} |E_k|_e$.
%   \end{theorem}

%  \footcite[][42]{Wheeden_Zygmund_2015}

% \documentclass[UTF8,a4paper,14pt]{article}
% \usepackage[utf8]{inputenc}
\usepackage{amsmath}
% \usepackage{algorithm,algorithmic}
\usepackage[linesnumbered,ruled,vlined]{algorithm2e}

% \usepackage{algorithmicx}
% \usepackage{algpseudocode}
\usepackage{hyperref}

% \usepackage{algpseudocode}
\usepackage{amssymb}
\usepackage{amsfonts}
%for 字體
%https://tug.org/FontCatalogue/
% \usepackage[T1]{fontenc}
% \usepackage{tgbonum}
% \usepackage[bitstream-charter]{mathdesign}
% \usepackage[T1]{fontenc}
% \usepackage{bm}#粗體
%\usepackage{boondox-calo}
\usepackage{textcomp}
\usepackage{fancyhdr}%导入fancyhdrf包
\usepackage{ctex}%导入ctex包
\usepackage{enumitem} %for在Latex使用條列式清單
\usepackage{varwidth}
\usepackage{soul} %for \ul
\usepackage{comment}%\begin{comment}\end{comment}
\usepackage{cancel}%\cancel{}
%\usepackage{unicode-math}

\usepackage[dvipsnames, svgnames, x_11names]{xcolor}

\usepackage[low-sup]{subdepth}
\usepackage{subdepth}

\newcommand{\indep}{\Perp \!\!\! \Perp}

\usepackage{amsthm}
\DeclareMathOperator{\R}{\mathbb{R}}
\DeclareMathOperator{\E}{\mathbb{E}}
\DeclareMathOperator{\Var}{\textbf{Var}}
\DeclareMathOperator{\Cov}{\textbf{Cov}}
\DeclareMathOperator{\Cor}{\textbf{Cor}}
\DeclareMathOperator{\X}{\mathbf{X}}
\DeclareMathOperator{\Pro}{\mathbf{P}}
\DeclareMathOperator{\M}{\mathbf{M}}
\DeclareMathOperator{\Id}{\mathbf{I}}
\DeclareMathOperator{\Y}{\mathbf{Y}}
\DeclareMathOperator{\MSFE}{\mathbf{MSFE}}
\DeclareMathOperator{\e}{\mathbb{e}}
\DeclareMathOperator{\V}{\mathbf{V}} 
\DeclareMathOperator{\tr}{\text{tr}}
\DeclareMathOperator{\A}{\textbf{A}}
\DeclareMathOperator{\diag}{diag}
\DeclareRobustCommand{\rchi}{{\mathpalette\irchi\relax}}
\newcommand{\irchi}[2]{\raisebox{\depth}{$#1\chi$}} % inner command, used by \rchi
\DeclareMathOperator*{\argmax}{arg\,max}
\DeclareMathOperator*{\argmin}{arg\,min}

\DeclareMathSizes{20}{10}{10}{5}

\usepackage[a4paper, margin=1in]{geometry}
% \setlength\parskip{5ex}% it would be better define distance in ex (5ex) 
                         %  or in pt, pc, mm, etc (see edit below)

\setlength{\parindent}{0pt}
\usepackage{array, makecell} %


%中英文設定
%\usepackage{fontspec}
% \setmainfont{TeX Gyre Termes}
% \usepackage{xeCJK} %引用中文字的指令集
% %\setCJKmainfont{PMingLiU}
% \setCJKmainfont{DFKai-SB}





% \setmainfont{Times New Roman}
% \setCJKmonofont{DFKai-SB}
\pagenumbering{arabic}%设置页码格式
\pagestyle{fancy}
\fancyhead{} % 初始化页眉
\usepackage{advdate}

% \newcommand{\yesterday}{{\AdvanceDate[-1]\today}}

\fancyhead[C]{Real Analysis II \quad HW  01\quad  R10A21126\quad  WANG YIFAN\quad   \today}
%\fancyhead[LE]{\textsl{\rightmark}}
%\fancyfoot{} % 初始化页脚
%\fancyfoot[LO]{奇数页左页脚}
%\fancyfoot[LE]{偶数页左页脚}
%\fancyfoot[RO]{奇数页右页脚}
%\fancyfoot[RE]{偶数页右页脚}

% \title{{Econometrics HW 05}}
% \author{R10A21126}
% \date{\today}

%\fancyhf{}
\usepackage{lastpage}
\cfoot{Page \thepage \hspace{1pt} of\, \pageref{LastPage}}

\renewcommand{\headrulewidth}{0.1pt}%分隔线宽度4磅
%\renewcommand{\footrulewidth}{4pt}

\allowdisplaybreaks
\usepackage[english]{babel}
%\usepackage{amsthm}
\newtheorem{theorem}{Theorem}[section]
\newtheorem{corollary}{Corollary}[theorem]
\newtheorem{lemma}[theorem]{Lemma}


\usepackage[most]{tcolorbox}

\definecolor{babyblue}{rgb}{0.54, 0.81, 0.94}

\newtcolorbox[auto counter]{mybox}[1]{
  % Define a new tcolorbox style with custom paragraph spacing
  before upper={\parskip=10pt},
    after upper={\parskip=10pt},
    enhanced,
    arc= 1 mm,boxrule=1.5pt,
    colframe=babyblue!80!pink,
    colback=white,
    coltitle=black,
    % colback=blue!5!white,
    attach boxed title to top left=
    {xshift=1.5em,yshift=-\tcboxedtitleheight/2},
    boxed title style={size=small,
    % frame hidden,
    colback=White},
    top=0.15in,
    % fonttitle=\bfseries,
    title= {#1},
    breakable
  }

\newtcolorbox[auto counter]{Problem}[2][]{
    enhanced,drop shadow={Pink!50!white},
    colframe=pink!80!white,
    fonttitle=\bfseries,
    title=Problem ~\thetcbcounter. #2,
    %separator sign={.},
    coltitle=black,
    colback=pink!15,
    top=0.15in,
    breakable
  }

\newenvironment{solution}
  {\renewcommand\qedsymbol{$\blacksquare$}\begin{proof}[Solution]}
  {\end{proof}}

\theoremstyle{definition}
\newtheorem{definition}{Definition}[section]

%\theoremstyle{notation}
\newtheorem*{notation}{\underline{Notation}}
%\newtheorem*{convention}{\underline{Convention}}
\newtheorem*{convention}{\underline{Convention}}

\theoremstyle{remark}
\newtheorem*{remark}{Remark}

\newenvironment{amatrix}[2]{%% [2] for 2 parameters 
  \left[\begin{array}
    %{cc\,|\,cc}
    %  {@{}*{#2}{c}\,|\,c*{#1}{c}}
     {{}*{#1}{c}\,|\,c*{#2}{c}}
}{%
  \end{array}\right]
}
% For augmented matrix  
%https://tex.stackexchange.com/questions/2233/whats-the-best-way-make-an-augmented-coefficient-matrix


% defines the paragraph spacing
\setlength{\parskip}{0.5em}


\usepackage[sorting=none, citestyle=verbose-inote,backref=true,ibidtracker=context,mincrossrefs=99,backend=biber, 
url = false,
doi = false, isbn=false,]{biblatex}

\addbibresource{R10A21126.bib}

\usepackage{graphicx}
\graphicspath{ {images/} }
\usepackage{caption}

% global change
\SetKwInput{KwData}{Input}
\SetKwInput{KwResult}{Output}
% https://tex.stackexchange.com/questions/299771/how-do-i-rename-data-from-kwdata-and-result-from-kwresult-in-begi

\hypersetup{hidelinks}

\let\oldemptyset\emptyset
\let\emptyset\varnothing
% https://tex.stackexchange.com/questions/22798/nice-looking-empty-set


\usepackage{pgfplots}

\begin{document}



\begin{Problem}[]{}
  Prove that \( L^\infty(E) \) is not separable for any \( E \) with \( |E| > 0 \).


\end{Problem}


\begin{solution}\,

Take $E = [0,1]$ for example. 

Suppose $L^\infty(E)$ is separable. 

Then there exists a dense subset $A$ consisting of countable elements, i.e., $A \subset L^\infty(E)$.

Define $f_\alpha = \rchi_{[0,\alpha]}$ for $\alpha \in [0,1]$, where $\rchi_{[0,\alpha]}$ is the indicator function for the interval $[0,\alpha]$. 

If $\alpha \neq \beta$, then $\|f_\alpha - f_\beta\|_\infty = 1$. 

Since $A$ is dense, for every $\alpha \in [0,1]$, and for every $\epsilon > 0$,

there exists $g_\alpha \in A$ such that 

% $$\|g_\alpha - f_\alpha\|_\infty < \frac{\epsilon}{4},$$ 
$$\|g_\alpha - f_\alpha\|_\infty < \epsilon.$$ 



By density, for $\beta \neq \alpha$, there is also $g_\beta \in A$ such that 

$$\|g_\beta - f_\beta\|_\infty < \epsilon.$$ 
% $$\|g_\beta - f_\beta\|_\infty \leq \frac{\epsilon}{4}.$$ 

Using the triangle inequality, we get
\[
\|f_\alpha - f_\beta\|_\infty \leq \|f_\alpha - g_\alpha\|_\infty + \|g_\alpha - g_\beta\|_\infty + \|g_\beta - f_\beta\|_\infty
\]
Rearranging, we have

\begin{equation*}
  \begin{aligned}
    \|g_\alpha - g_\beta\|_\infty &\geq \|f_\alpha - f_\beta\|_\infty - \|f_\alpha - g_\alpha\|_\infty - \|f_\beta - g_\beta\|_\infty\\
    &> 1 - \epsilon - \epsilon = 1 - 2\epsilon
  \end{aligned}
\end{equation*}

Since $\epsilon$ can be arbitrarily small, set $\epsilon = \frac{1}{4}$. 

This implies $\|g_\alpha - g_\beta\|_\infty > \frac{1}{2}$. Therefore, $g_\alpha \neq g_\beta$ if $\alpha \neq \beta$.

This implies that the function mapping $\alpha \mapsto g_\alpha$ from $[0,1]$ to $A$ is injective, i.e., "one-one." 

Since $[0,1]$ is uncountably infinite, this implies that $A$ is also uncountably infinite. 

However, $A$ is assumed to be a countable subset of $L^\infty[0,1]$. 

The existence of an uncountable subset $\{g_\alpha \mid \alpha \in [0,1]\}$ within $A$ contradicts the countability of $A$.

Therefore, $L^\infty(E)$ is not separable.

\end{solution}


\pagebreak

\begin{Problem}[]{}
  Let \( 1 \leq p < \infty \) and \( f \in L^p(\mathbb{R}^n) \). Show that the function \( g \) defined by
  \[
  g_f(h) = \|f(x + h) - f(x)\|_p
  \]
  is a uniformly continuous function on \(\mathbb{R}^n\). Is the same statement true when \( 0 < p < 1 \)?
  

\end{Problem}


The statement only holds for \(1 \leq p < \infty\).

\textbf{Theorem:} Continuous functions with compact support are dense in \(L^p\)

Let \(f \in L^p(\mathbb{R}^n)\). For every \(\epsilon > 0\), there exists a function \(k\) that is continuous with compact support such that:
\[
\|f - k\|_p < \epsilon.
\]


% \textbf{Using:} Continuous functions with compact support are dense in \(L^p\)

% % , which implies the function is uniformly continuous.

% Let \( f \in L^p(\mathbb{R}^n) \). Let \(k\) be the function that is continuous with compact support \(E\). For every \(\epsilon > 0\), there exists \(k\) such that:
% \[
% \|f(x) - k(x)\|_p < \epsilon.
% \]

% \textbf{Claim (Fact, not used):} The function defined as
% \begin{equation*}
%   \begin{aligned}
%     g_k(h) &= \|k(x+h) - k(x)\|_p
%   \end{aligned}
% \end{equation*}
% is uniformly continuous. This is just a fact, not used in this argument.

\dotfill

\textbf{We want to show \(g_f(h)\) is uniformly continuous:} i.e. For every \(\epsilon > 0\), there exists \(\delta\) such that:
\begin{equation*}
  \begin{aligned}
    |g_f(h_1) - g_f(h_2)| &< \epsilon, \quad \text{whenever } |h_1 - h_2| < \delta.
  \end{aligned}
\end{equation*}

We know that:
\begin{equation*}
  \begin{aligned}
    |g_f(h_1) - g_f(h_2)|   &:= |\|f(x+h_1) - f(x)\|_p - \|f(x+h_2) - f(x)\|_p| \\
    \text{(By Reverse Triangle Inequality) }& \leq \|f(x+h_1) - f(x+h_2)\|_p \\
    \text{(By Minkowski's Inequality) }&\leq \|f(x+h_1) - k(x+h_1)\|_p + \|k(x+h_1) - k(x+h_2)\|_p \\
    &\quad + \|f(x+h_2) - k(x+h_2)\|_p,
  \end{aligned}
\end{equation*}
where \(\forall \epsilon>0:\)

\begin{enumerate}
  \item 
  \begin{equation*}
  \begin{aligned}
    &\|f(x+h_1) - k(x+h_1)\|_p < \epsilon, \\
    &\|f(x+h_2) - k(x+h_2)\|_p < \epsilon,
  \end{aligned}
\end{equation*}
by the existence of a continuous function \(k\) with compact support dense in \(L^p\);

\item 
\begin{equation*}
  \begin{aligned}
    &\|k(x+h_1) - k(x+h_2)\|_p < \epsilon, \\
  \end{aligned}
\end{equation*}
\text{since \(k\) is uniformly continuous with compact support \(E\)}.
% \begin{equation*}
%   \begin{aligned}
%     \|k(x+h_1) - k(x+h_2)\|_p &= \left( \int_E |k(x+h_1) - k(x+h_2)|^p \, dx \right)^{1/p} \\
%     &< \left( \epsilon^p |E| \right)^{1/p} \\
%     &= \epsilon',
%   \end{aligned}
% \end{equation*}
% where \(|E| < \infty\).

\end{enumerate} 
  
% Thus,

Thus, the estimations above lead to:
\begin{equation*}
  \begin{aligned}
    \|f(x+h_1) - f(x+h_2)\|_p &\leq \|f(x+h_1) - k(x+h_1)\|_p + \|k(x+h_1) - k(x+h_2)\|_p \\
    &\quad + \|f(x+h_2) - k(x+h_2)\|_p \\
    &< \epsilon + \epsilon + \epsilon = 3\epsilon.
  \end{aligned}
\end{equation*}

Therefore, choosing \(\delta\) small enough to ensure the inequality above, whenever \(|h_1 - h_2| < \delta\), guarantees that:

\begin{equation*}
  \begin{aligned}
    |g_f(h_1) - g_f(h_2)| &< 3\epsilon.
  \end{aligned}
\end{equation*}

Hence, we have shown that \( g_f(h) \) is uniformly continuous. This confirms that any \( L^p \) function with \( 1 \leq p < \infty \) behaves such that the mapping \( h \mapsto \|f(x+h) - f(x)\|_p \) is uniformly continuous on \( \mathbb{R}^n \).

\textbf{For \(0 < p < 1\):} 

Minkowski's inequality fails for \(0 < p < 1\). 

To see this, take \(E = (0, 1)\), \(f = \chi_{(0, \frac{1}{2})}\), and \(g = \chi_{(\frac{1}{2}, 1)}\). 

Then \(\|f+g\|_p = 1\), while \(\|f\|_p + \|g\|_p = 2^{-\frac{1}{p}} + 2^{-\frac{1}{p}} = 2 \cdot 2^{-\frac{1}{p}} = 2^{1-\frac{1}{p}} < 1\).

This demonstrates that the statement does not hold for \(0 < p < 1\).
% The behavior of \( L^p \) spaces when \( p < 1 \) differs significantly as these spaces do not form a normed space, and the usual properties like the triangle inequality (and hence, Minkowski's inequality) do not hold. Without these fundamental properties, demonstrating uniform continuity in this manner is not feasible, implying the uniform continuity of \( g_f(h) \) may not hold when \( p < 1 \).


\end{document}