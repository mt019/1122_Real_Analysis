\documentclass[UTF8,a4paper,10pt]{article}

% \begin{equation*}
%   \begin{aligned}
%   \end{aligned}
% \end{equation*}

% \begin{mybox}{}
% \end{mybox}


% \begin{Problem}[]{}
% \end{Problem}

% \begin{solution}\,
% \end{solution}
  

% \begin{enumerate}[label=(\alph*)]
% \end{enumerate} 

% \setcounter{section}{3} 
% \setcounter{theorem}{3}

% \begin{theorem}\label{thm:3.4}
%   If $E = \bigsqcup_{k} E_k$ is a countable union of sets, then $|E|_e \leq \sum_{k} |E_k|_e$.
%   \end{theorem}

%  \footcite[][42]{Wheeden_Zygmund_2015}

% \documentclass[UTF8,a4paper,14pt]{article}
% \usepackage[utf8]{inputenc}
\usepackage{amsmath}
% \usepackage{algorithm,algorithmic}
\usepackage[linesnumbered,ruled,vlined]{algorithm2e}

% \usepackage{algorithmicx}
% \usepackage{algpseudocode}
\usepackage{hyperref}

% \usepackage{algpseudocode}
\usepackage{amssymb}
\usepackage{amsfonts}
%for 字體
%https://tug.org/FontCatalogue/
% \usepackage[T1]{fontenc}
% \usepackage{tgbonum}
% \usepackage[bitstream-charter]{mathdesign}
% \usepackage[T1]{fontenc}
% \usepackage{bm}#粗體
%\usepackage{boondox-calo}
\usepackage{textcomp}
\usepackage{fancyhdr}%导入fancyhdrf包
\usepackage{ctex}%导入ctex包
\usepackage{enumitem} %for在Latex使用條列式清單
\usepackage{varwidth}
\usepackage{soul} %for \ul
\usepackage{comment}%\begin{comment}\end{comment}
\usepackage{cancel}%\cancel{}
%\usepackage{unicode-math}

\usepackage[dvipsnames, svgnames, x_11names]{xcolor}

\usepackage[low-sup]{subdepth}
\usepackage{subdepth}

\newcommand{\indep}{\Perp \!\!\! \Perp}

\usepackage{amsthm}
\DeclareMathOperator{\R}{\mathbb{R}}
\DeclareMathOperator{\E}{\mathbb{E}}
\DeclareMathOperator{\Var}{\textbf{Var}}
\DeclareMathOperator{\Cov}{\textbf{Cov}}
\DeclareMathOperator{\Cor}{\textbf{Cor}}
\DeclareMathOperator{\X}{\mathbf{X}}
\DeclareMathOperator{\Pro}{\mathbf{P}}
\DeclareMathOperator{\M}{\mathbf{M}}
\DeclareMathOperator{\Id}{\mathbf{I}}
\DeclareMathOperator{\Y}{\mathbf{Y}}
\DeclareMathOperator{\MSFE}{\mathbf{MSFE}}
\DeclareMathOperator{\e}{\mathbb{e}}
\DeclareMathOperator{\V}{\mathbf{V}} 
\DeclareMathOperator{\tr}{\text{tr}}
\DeclareMathOperator{\A}{\textbf{A}}
\DeclareMathOperator{\diag}{diag}
\DeclareRobustCommand{\rchi}{{\mathpalette\irchi\relax}}
\newcommand{\irchi}[2]{\raisebox{\depth}{$#1\chi$}} % inner command, used by \rchi
\DeclareMathOperator*{\argmax}{arg\,max}
\DeclareMathOperator*{\argmin}{arg\,min}

\DeclareMathSizes{20}{10}{10}{5}

\usepackage[a4paper, margin=1in]{geometry}
% \setlength\parskip{5ex}% it would be better define distance in ex (5ex) 
                         %  or in pt, pc, mm, etc (see edit below)

\setlength{\parindent}{0pt}
\usepackage{array, makecell} %


%中英文設定
%\usepackage{fontspec}
% \setmainfont{TeX Gyre Termes}
% \usepackage{xeCJK} %引用中文字的指令集
% %\setCJKmainfont{PMingLiU}
% \setCJKmainfont{DFKai-SB}





% \setmainfont{Times New Roman}
% \setCJKmonofont{DFKai-SB}
\pagenumbering{arabic}%设置页码格式
\pagestyle{fancy}
\fancyhead{} % 初始化页眉
\usepackage{advdate}

% \newcommand{\yesterday}{{\AdvanceDate[-1]\today}}

\fancyhead[C]{Real Analysis II \quad HW  01\quad  R10A21126\quad  WANG YIFAN\quad   \today}
%\fancyhead[LE]{\textsl{\rightmark}}
%\fancyfoot{} % 初始化页脚
%\fancyfoot[LO]{奇数页左页脚}
%\fancyfoot[LE]{偶数页左页脚}
%\fancyfoot[RO]{奇数页右页脚}
%\fancyfoot[RE]{偶数页右页脚}

% \title{{Econometrics HW 05}}
% \author{R10A21126}
% \date{\today}

%\fancyhf{}
\usepackage{lastpage}
\cfoot{Page \thepage \hspace{1pt} of\, \pageref{LastPage}}

\renewcommand{\headrulewidth}{0.1pt}%分隔线宽度4磅
%\renewcommand{\footrulewidth}{4pt}

\allowdisplaybreaks
\usepackage[english]{babel}
%\usepackage{amsthm}
\newtheorem{theorem}{Theorem}[section]
\newtheorem{corollary}{Corollary}[theorem]
\newtheorem{lemma}[theorem]{Lemma}


\usepackage[most]{tcolorbox}

\definecolor{babyblue}{rgb}{0.54, 0.81, 0.94}

\newtcolorbox[auto counter]{mybox}[1]{
  % Define a new tcolorbox style with custom paragraph spacing
  before upper={\parskip=10pt},
    after upper={\parskip=10pt},
    enhanced,
    arc= 1 mm,boxrule=1.5pt,
    colframe=babyblue!80!pink,
    colback=white,
    coltitle=black,
    % colback=blue!5!white,
    attach boxed title to top left=
    {xshift=1.5em,yshift=-\tcboxedtitleheight/2},
    boxed title style={size=small,
    % frame hidden,
    colback=White},
    top=0.15in,
    % fonttitle=\bfseries,
    title= {#1},
    breakable
  }

\newtcolorbox[auto counter]{Problem}[2][]{
    enhanced,drop shadow={Pink!50!white},
    colframe=pink!80!white,
    fonttitle=\bfseries,
    title=Problem ~\thetcbcounter. #2,
    %separator sign={.},
    coltitle=black,
    colback=pink!15,
    top=0.15in,
    breakable
  }

\newenvironment{solution}
  {\renewcommand\qedsymbol{$\blacksquare$}\begin{proof}[Solution]}
  {\end{proof}}

\theoremstyle{definition}
\newtheorem{definition}{Definition}[section]

%\theoremstyle{notation}
\newtheorem*{notation}{\underline{Notation}}
%\newtheorem*{convention}{\underline{Convention}}
\newtheorem*{convention}{\underline{Convention}}

\theoremstyle{remark}
\newtheorem*{remark}{Remark}

\newenvironment{amatrix}[2]{%% [2] for 2 parameters 
  \left[\begin{array}
    %{cc\,|\,cc}
    %  {@{}*{#2}{c}\,|\,c*{#1}{c}}
     {{}*{#1}{c}\,|\,c*{#2}{c}}
}{%
  \end{array}\right]
}
% For augmented matrix  
%https://tex.stackexchange.com/questions/2233/whats-the-best-way-make-an-augmented-coefficient-matrix


% defines the paragraph spacing
\setlength{\parskip}{0.5em}


\usepackage[sorting=none, citestyle=verbose-inote,backref=true,ibidtracker=context,mincrossrefs=99,backend=biber, 
url = false,
doi = false, isbn=false,]{biblatex}

\addbibresource{R10A21126.bib}

\usepackage{graphicx}
\graphicspath{ {images/} }
\usepackage{caption}

% global change
\SetKwInput{KwData}{Input}
\SetKwInput{KwResult}{Output}
% https://tex.stackexchange.com/questions/299771/how-do-i-rename-data-from-kwdata-and-result-from-kwresult-in-begi

\hypersetup{hidelinks}

\let\oldemptyset\emptyset
\let\emptyset\varnothing
% https://tex.stackexchange.com/questions/22798/nice-looking-empty-set


\usepackage{pgfplots}

\begin{document}



\begin{Problem}[]{}
 Let $f : \mathbb{R} \rightarrow \mathbb{C}$. Prove that $f$ satisfies the Lipschitz condition
  \[
  |f(x) - f(y)| \leq M|x - y|
  \]
  for some $M > 0$ and for all $x, y \in \mathbb{R}$, if and only if $f$ satisfies the following two properties:
  \begin{enumerate}
      \item[(i)] $f$ is absolutely continuous.
      \item[(ii)] $|f'(x)| \leq M$ for a.e. $x$
  \end{enumerate}
  
\end{Problem}

\(\Rightarrow \)

Suppose \(f\) is Lipschitz continuous. For \(\epsilon > 0\), let \(\delta = \frac{\epsilon}{M}\), whenever \( \sum |b_i - a_i|< \delta\), we have

\[\sum |f(b_i) - f(a_i)| \leq M \sum |b_i - a_i| < \epsilon,\]

implying that \(f\) is absolutely continuous.

By Theorem 7.27 and Corollary 7.23, we can see that \(f'\) exists (\(f\) is differentiable) almost everywhere.

For \(x\) where \(f'(x)\) exists, the Lipschitz condition implies that, for all \(h\),

\begin{equation*}
  \begin{aligned}
    \left| \frac{f(x+h) - f(x)}{h} \right| \leq M
  \end{aligned}
\end{equation*}

Taking the limit as \(h\to 0\), we have

\begin{equation*}
  \begin{aligned}
    \left|f'(x) \right|  = \lim_{h\to 0}\left| \frac{f(x+h) - f(x)}{h} \right| \leq M
  \end{aligned}
\end{equation*}

\dotfill

\(\Leftarrow \)

Suppose that \(f\) is absolutely continuous. By Theorem 7.27 and Corollary 7.23, we have that \(f\) is of bounded variation and thus \(f'\) exists almost everywhere.

By Theorem 7.29, for all \(x < y \in \mathbb{R} \),

\[\left| f(x) - f(y) \right| = \left| \int_{x}^{y} f'(t) \, dt \right| \leq \int_{x}^{y} \left| f'(t) \right| \, dt \leq \int_{x}^{y}  M \, dt = M \left| x - y \right|.
\]

We can conclude that \( f \) is Lipschitz continuous.

\begin{mybox}{Corollary 7.23}
  If \( f \) is of bounded variation on \([a, b]\), then \( f' \) exists almost everywhere in \([a, b]\), and \( f' \in L[a, b] \).
\end{mybox}

\begin{mybox}{Theorem 7.27}

  If \( f \) is absolutely continuous on \([a, b]\), then it is of bounded variation on \([a, b]\).

\end{mybox}


\begin{mybox}{Theorem 7.29}

A function \( f \) is absolutely continuous on \([a, b]\) if and only if \( f' \) exists almost everywhere in \([a, b]\), \( f' \) is integrable on \([a, b]\), and for \( a \leq x \leq b \),
\[
f(x) - f(a) = \int_{a}^{x} f'(t) \, dt.
\]
\end{mybox}

\pagebreak

\begin{mybox}{}
    
  A function \( \phi \) is convex in \( (a, b) \) if and only if
  \[
    \phi(\theta x_1 + (1 - \theta) x_2) \leq \theta \phi(x_1) + (1 - \theta) \phi(x_2)
    \]
    for \( x_1 , x_2 \in (a,b)\) and \( 0 \leq \theta \leq 1 \). 
    
  \end{mybox}

  \begin{mybox}{Theorem 7.40}


 If \(\phi\) is convex in (a, b), then \(\phi\) is continuous in (a, b). Moreover,
\(\phi\) exists except at most in a countable set and is monotone increasing.

\end{mybox}

\begin{Problem}[]{}
  Prove that $f$ is convex on $(a, b)$ if and only if it is continuous and
  \[
  f\left(\frac{x + y}{2}\right) \leq \frac{f(x) + f(y)}{2}
  \]
  for all $x, y \in (a, b)$.
  

\end{Problem}

\(\Rightarrow \)

Suppose that \(f\) is convex. Following the definition of convexity, let \(\theta = \frac{1}{2}\), we have

\[
  f\left(\frac{x + y}{2}\right) \leq \frac{f(x) + f(y)}{2}
  \]
  for all $x, y \in (a, b)$.
  
By Theorem 7.40, \(f\) is continuous.

\dotfill

\(\Leftarrow \)

% Suppose that \(f\) is continuous and the inequality holds.

% Let a sequenence \(d_k = \frac{m}{2^n}\), where \(n \in\mathbb{N}\), and \(m = 0,1,\ldots,2^n\).

% For all \(d_k\), we have 

% \[
%   f\left(d_k a + (1-d_k) b\right) 
%    \leq d_k f(a) + (1-d_k) f(b)
%   \]

% For any \(\theta \in [0,1]\), we can find \(\lim_{k\to \infty}d_k = \theta\).





% \dotfill

Suppose that \( f \) is continuous and satisfies the midpoint inequality for all \( x, y \in (a, b) \).

To prove convexity, consider any \( x, y \in (a, b) \) and any \( \theta \in [0, 1] \). We need to show that
\[
f(\theta x + (1-\theta) y) \leq \theta f(x) + (1-\theta) f(y).
\]

This can be proved using an induction approach on the dyadic rationals (i.e., numbers of the form \( \frac{m}{2^n} \), where \(n \in\mathbb{N}\), and \(m = 0,1,\ldots,2^n\)), then generalizing to all \( \theta \) using the continuity of \( f \).

\textit{Base Case:} For \( n=1 \), the midpoint inequality ensures that the convexity condition holds for \( \theta = \frac{1}{2} \).

% \textit{Inductive Step:} Assume the convexity condition holds for all dyadic rationals of the form \( \frac{m}{2^n} \) for some \( n \). Consider \( \theta = \frac{m}{2^{n+1}} \) or \( \theta = 1 - \frac{m}{2^{n+1}} \). Through a detailed calculation using the inductive hypothesis, this can be shown to satisfy the convexity condition.

\textit{Inductive Step:} Assume the condition holds for \( n \). Consider \( \theta = \frac{m}{2^{n+1}} \).

\textit{Case 1:} If \( m \) is even, \( \theta \) is a dyadic rational of the form \( \frac{k}{2^n} \), so the hypothesis applies.

\textit{Case 2:} If \( m \) is odd, write \( \theta \) as
\[
\theta = \frac{1}{2} \left(\frac{(m-1)/2}{2^n}\right) + \frac{1}{2} \left(\frac{(m+1)/2}{2^n}\right),
\]
noting that \(\frac{m-1}{2}\) and \(\frac{m+1}{2}\) are integers.
% both terms being dyadic rationals of order \( n \).

Using the midpoint inequality:
\[
f\left(\theta x + (1-\theta) y\right) \leq \frac{1}{2} f\left(\frac{(m-1)/2}{2^n} x + \left(1 - \frac{(m-1)/2}{2^n}\right) y\right) + \frac{1}{2} f\left(\frac{(m+1)/2}{2^n} x + \left(1 - \frac{(m+1)/2}{2^n}\right) y\right).
\]
By applying the convexity condition assumed for \( \frac{(m-1)/2}{2^n} \) and \( \frac{(m+1)/2}{2^n} \):
\begin{equation*}
  \begin{aligned}
f\left(\theta x + (1-\theta) y\right) &\leq \frac{1}{2} \left(\frac{(m-1)/2}{2^n} f(x) + \left(1 - \frac{(m-1)/2}{2^n}\right) f(y)\right) + \frac{1}{2} \left(\frac{(m+1)/2}{2^n} f(x) + \left(1 - \frac{(m+1)/2}{2^n}\right) f(y)\right) \\
&= \theta f(x) + (1-\theta) f(y).
\end{aligned}
\end{equation*}

Thus, the convexity condition is preserved for \( n+1 \).


The continuity of \( f \) implies that since the inequality holds for all dyadic rationals, it also holds by the limit for any \( \theta \in [0, 1] \) as dyadic rationals are dense in \([0,1]\).

Thus, \( f \) is convex on \((a, b)\).


\end{document}